

\documentclass[3p,times]{elsarticle}

\usepackage{ecrc}


\volume{00}


\firstpage{1}


\journalname{Procedia Computer Science}


\runauth{}


\jid{procs}


\jnltitlelogo{Procedia Computer Science}


\CopyrightLine{2011}{Published by Elsevier Ltd.}


\usepackage{amssymb}

\usepackage[figuresright]{rotating}



\begin{document}

\begin{frontmatter}



\dochead{Karinthy}


\title{Less-known gems of a genius}



\author{Eszter Domokos-Kovari}

\address{University of Sopron - PhD student}

\begin{abstract}
This article contains some less-know but gorgeous and witty works of the genius author, Frigyes Karinthy.
\end{abstract}

\begin{keyword}
Karinthy \sep humour \sep Nyugat \sep Kosztolányi



\end{keyword}

\end{frontmatter}


\section{Dana Idák*}
\label{}

Lent a lenső lélekuton \{Hádesz öblén, [hol a lélek élet – állott (élet – állott, mint az állat) s aspodélosz illatokkal] öblögetve, ablakokba\} ablakokba öblögetve, öblögekbe ablakogva és makogva mekegve.\\


És mekegve és makogva lent a mélybe, lent $ \sqrt[2]{lent}  $ az éjbe, hol a kéjbe $ {ejbe,melybe \choose   kejbe,ejbe} $.

Ötven asszony, logarithmus ötven asszony és emelve és kivonva, négyzetekre köbgyökökre, ötven asszony, hetven asszony, százhuszonhét bűnös asszony, ötven órjás amphorába, asfodélosz ötven asszony = bünös asszony, mennyi jött ki, mennyi jött ki.\\
Százkilencven bűnös asszony, óriási amphorába, amphorába \begin{math} \sqrt[2]{amphoraba} \end{math} , rába, rába, rába, majd mekegve, $(\log) $ majd makogva, mindörökre, mindhiába, (mert hiába [mind hiába] töltögetve, öblögetve) öblögetve ablakokba, ablagokva, és makogva,\begin{displaymath} (oblogekbe)^{2} \end{displaymath} és mekegve és makogva, fogjanak $ \sqrt[2]{fogjanak} $ meg, fogjanak meg…\\

\begin{center}
(Megfogják.)\cite{karinthy_dana_1912}
\end{center}

\section{Struggle for life}

Pajtás, úgy fest, alulmaradtál
A Tétel és Törvény szerint –
Dögödre már hiéna szaglász
S a varjú éhesen kering.
Nem is a falka volt erősebb
Apró vadak tángáltak el –
S hogy írhádon ki osztozik most
Véreb? Veréb? Nem érdekel.
Öklöd, mikor lecsapni kellett
Mindig megállt a féluton –
Jóság volt? Gyöngeség? Nem értem.
Félsz? Gőg? Szemérem? Nem tudom.
Talán csak undor. Jól van így is.
Megnyugszom. Ámen. Úgy legyen.
Inkább egyenek meg a férgek
Minthogy a férget megegyem. \cite{karinthy_struggle_1938}

\footnote{Matematikai költemény.}

\section{Kínai vers}

Tea

Csendben teázom
Előttem findzsa, kancsó
Reád gondolok
Ó Indzsamandzsó (*)
\\
Atyák
(Bet.Len, nagy oj-jé, de mekkora költő a Kr. (Károlyi) előtti Bung-dinasztiából.)
\\
Én is füstöltem,
Most ő pipázik
Az utódom
Semmi sem változik
Füst az élet
\\
Kertben
(Haj-Hagy-Ma, kínai költő, a Link-dinasztia korából. Kr. e. 1425 febr. 9-én, délután félhatkor.)

Sétálok kis kertemben
A cseresnyefa virágzik
Egyik lábam a másik előtt
A másik az egyik után
Jó hogy nincs harmadik
Nem tudnám hová tegyem
Vannak ilyen örömök az életben

Ez csak egyszerű? \cite{karinthy_kinai_nodate}

\section{Kosztolányi Dezső}

Kinek meséljem el, hogy annak lássa, aminek látni kell – kit érdekelhetne mohóbban, szilajabban, okosabban
és nemesebben, mint Téged, Didus, a temetésed, amiről megígértem, hogy beszámolok? Mert próbáltam erre gondolni és
amarra az élők közül és kisült, hogy mindig újra az jön ki belőle, ahogy ott állok, koporsód mellett a könnyezők között: 
mit szólnál, Didus, a temetésedhez? Mert ez már, úgy látszik, nem megy másképpen, s ez már így lesz velem, egy életen át, 
nem segít rajta, hogy meghaltál: ha valami nagyszerűen szépet, vagy nagyszerűen szomorút látok, élek át, vagy akárcsak hallok
vagy olvasok – gépiesen jelentkezik a kérdés, mit szólna ehhez Desiré? Aminthogy, ugye, Nálad is sokszor volt ez így: 
ezt el kell mesélnem Fricinek. Hiszen egy életen át játszottunk, mindig játszottunk, izgatottan és lelkendezve,
a Szerepet játszottuk ezer álarcban, mélyebb szenvedéllyel, mint a tulajdon magánéletünk komédiáját.\cite{karinthy_kosztolanyi_1936}



\bibliographystyle{elsarticle-num}
\bibliography{references}



\end{document}

